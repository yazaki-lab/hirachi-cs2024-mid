\documentclass[dvipdfmx]{cs-handout}

%Font Info
\usepackage[T1]{fontenc}
\usepackage{otf}
\usepackage{color}
\usepackage[dvips]{graphicx}
\usepackage{latexsym}
\usepackage{listings,jvlisting}
\usepackage{url}
\usepackage{here}
\usepackage{paralist}

\newcommand{\Note}[1]{\noindent \textbf{\textcolor{blue}{#1}}}

\DeclareRobustCommand{\inlinelist}[1]{\begin{inparaenum}[(1)] #1 \end{inparaenum}}

\def\Underline{\setbox0\hbox\bgroup\let\\\endUnderline}
\def\endUnderline{\vphantom{y}\egroup\smash{\underline{\box0}}\\}
\def\|{\verb|}

\lstset{
  basicstyle={\ttfamily},
  identifierstyle={\small},
  commentstyle={\smallitshape},
  keywordstyle={\small\bfseries},
  ndkeywordstyle={\small},
  stringstyle={\small\ttfamily},
  frame={tb},
  breaklines=true,
  columns=[l]{fullflexible},
  numbers=left,
  xrightmargin=0zw,
  xleftmargin=3zw,
  numberstyle={\scriptsize},
  stepnumber=1,
  numbersep=1zw,
  lineskip=-0.5ex
}
\renewcommand{\lstlistingname}{List}
\def\newblock{\hskip .11em plus .33em minus .07em}


%Document Info
\title{柔軟で動的な実行環境を提供するための\\コンテナ型仮想化基盤アーキテクチャの検討}
\author{平地浩一}
\stdnum{2110525}
\office{矢崎研究室}
%\lhead{20YY年度 情報理工学研究科 情報・ネットワーク工学専攻 CSプログラム 修士論文(中間)発表}
\lhead{2024年度 情報理工学域 I類 CSプログラム 卒業研究(中間)発表}
\rhead{2024/10/3}

\begin{document}
\maketitle

\section{はじめに}
%\Note{研究ストーリー(このフォントはコメント)}

教育,研究,開発においてユーザ専用の隔離された実行環境が必要とされる.
しかし,共用マシンでは管理者によるリソースやアプリケーションの制限がある.
ユーザ専用の環境を提供することで,管理者の制約なく自由に環境を利用できるようになる.
一般的には物理マシンや仮想マシンを割り当てるが,リソースが有限であり,大人数に割り当て続けるのは現実的ではない.
これを解決するため,最適なリソース分配を動的に行う必要がある.
仮想化の粒度に応じた様々な仮想化技術があるが,その中でもコンテナ仮想化技術に注目した.
コンテナはOSカーネル上で実行環境を分離し,物理マシンの仮想化よりも細かく制御できる利点がある.

本研究では,利用者が自分自身で様々なコンピューティング環境をオンデマンドに使用できるコンテナ型仮想化基盤を提案・構築する.

\section{関連研究}

\subsection{ContainerSSH}

ContainerSSHは,ユーザからのSSH接続に対してKubernetesやDockerなどのバックエンドを介して動的にコンテナを立ち上げる仕組みを提供するOSSである\cite{ContainerSSH}.
ユーザは管理者が予めコンテナイメージを選択することはできない.

\subsection{sshr}

sshrは,鶴田氏らによって開発されたSSH Proxyサーバである\cite{Tsuruta2020}.
sshrでは,大規模なサーバ群においてユーザがSSH接続に用いたユーザIDに応じて,予め紐づけられたサーバへの接続を中継する.
ユーザ自身が接続するサーバを選択することはできない.

\section{提案するシステムの構成}

本研究で提案するシステムの構成を図\ref{fig:system}に示す.
本研究では主に,図\ref{fig:system}中におけるControllerに該当する機能を実装する.
具体的には,図\ref{fig:diff}に示すように既存のContainerSSHのControllerの実装に変更を加える.
%
\begin{figure}[htbp]
\includegraphics[width=\linewidth]{./fig/simple.png}
%キャプションでもっと長めの説明をつける(図とキャプションを見ただけでわかるような)
%図の中で使われているような,見ただけではわからないものに関しては書いておいた方が良い
\caption{提案システムの構成}
\label{fig:system}
\end{figure}
%
\begin{figure}[tb]
\includegraphics[width=\linewidth]{./fig/diff.png}
\caption{既存のContainerSSH(左)と提案システム(右)のController実装の違い}
\label{fig:diff}
\end{figure}
%
システムは,ユーザーに対して実行環境毎を異なるエンドポイントが提供する.
ユーザはSecure SHell(SSH)やRemote Desktop Protocol(RDP)といったプロトコルによってシステムのInterfaceへ接続すると,Controllerへユーザが要求する環境が伝達される.
Controllerでは,KubernetesやDockerといったバックエンドをAPI経由で呼び出し,要求されたコンピューティング環境を動的に提供する.

\section{既存システムのレイテンシ測定実験}

実装するにあたり,ContainerSSHの実装を参考とする.
提案するシステムに要求される時間的な制約を確認するため,ContainerSSHを用いて複数人が同時にアクセスした際のレイテンシを測定した.
% (コメント:なんでこんな実験をするのか,何を確認したくて実験をするのかを書く.以下は適当に書いたので直してください.)

実験では,backendにDockerを用いた.
検証環境においてContainerSSHとDockerを実行し, sshコマンドで同時接続を行なった際の接続時間を計測した.

% Comment: クライアント側から応答時間を測ったことがわかるタイトル
\subsection{クライアント側からSSH接続時の応答時間の計測}

まず初めに複数のユーザが同時に接続した際に,ユーザがSSH接続をしてからコマンドを実行して終了するまでの応答時間の変化を計測した.
スクリプトを用いて同時に接続する数を1から10づつ100まで増やしていき,接続数ごとの処理時間を記録した.
結果を図\ref{fig:connect}に示す.
横軸は並列接続数で,縦軸は要した時間を表している.
%
%図をどうやって見るのかの説明(横軸と縦軸で,線が何を表しているのか,凡例を入れる)
%このグラフの何を見て欲しいのか,読み取れる事実のうち,注目して欲しいところを説明する.
%接続数が少ないうちには,ある程度一定だが,接続数が増えていくほど傾きが急になるような傾向が読み取れる
%事実だけを書く
%
%補助線を入れて,横のラインとわかるやすくする
%メモリの線を図の内側に入れるようにする
%0,0と100あたりの余白をなくすようにする
%0,0を消してしまった方が良い
%
%一番早い人と一番遅い人と平均
\begin{figure}[tb]
\includegraphics[width=0.9\linewidth]{./fig/connect.png}
\caption{並列実行数と接続時間の関係}
\label{fig:connect}
\end{figure}
% Comment: 接続にかかった時間 をもっと精密に SSHコマンドを実行してから接続が成功し,dateコマンドを実行したのち,ssh接続を切断するまでの時間を,クライアント側からtimeコマンドで測定した時間

% Comment: 何を測ったのかわかるようなタイトル
\subsection{サーバ側における各部分処理にかかる時間の測定}

前の実験より,ユーザがSSHコマンドを実行してからの応答時間は同時接続数に応じて長くなることがわかった.
ここでは,どの処理によって接続時間が増加するのかを検証した.

%
\begin{figure}[tb]
\includegraphics[width=\linewidth]{./fig/step.png}
\caption{ContainerSSHにおける内部処理}
\label{fig:step}
\end{figure}
%

%セッションごとに一連の流れ,全体の処理を,各部分処理に分けて時間を測定

実験では,並列に接続するクライアント数を1から100まで変化させ,図\ref{fig:step}に示す各処理にかかる平均の時間を算出した.
結果を図\ref{fig:detail}に示す.横軸は並列接続数で,縦軸は各処理の所用時間を積み上げて示している.
%
%その他の除外している部分も入れた方が良い
%グラフの読み取りの説明も追加する
%スクリプト化して何回か回してみる
%コンテナを消す方が時間かかっている,authとsshの認証が時間がかかってる
%基本的には接続数が増えるほど処理時間が増えていく(全てのパートにおいて)
%認証とSSH接続の部分が支配的になっている
\begin{figure}[tb]
\includegraphics[width=0.9\linewidth]{./fig/detail.png}
\caption{並列実行数と主な処理時間の関係}
\label{fig:detail}
\end{figure}
%
% Comment: 接続にかかった時間 をもっと精密に SSHコマンドを実行してから接続が成功し,dateコマンドを実行したのち,ssh接続を切断するまでの時間を,クライアント側からtimeコマンドで測定した時間
% Comment: sshの接続にかかった最長の時間

\section{考察}

%ノートパソコンぐらいの接続でも100同時接続
%スケールアップでどこまで頑張れるのかを見る(スケールアウトする前に)
%独自の認証システムなどを使うと


図\ref{fig:connect}より,ContainerSSHでは同時接続数が多くなるほど1つの接続にかかる時間が増加していくことが確認できた.
同時接続数が100程度の場合,最長で22秒ほどの時間がかかっており,ユーザがストレスなく現実的に利用できるとは言えない.
平均の接続時間に関しても100接続であっても約14秒程度かかっている.
より接続数が増えた場合,これらの時間はさらに増加していくと考える.
提案システムが想定している200〜300名程度の同時利用を考えると,ユーザ数に応じて顕著に処理時間が増加する部分においては対策が必要である.

%最大と最小を確認した上でどのように書くのかは後から考える

図\ref{fig:detail}を参照すると,ContainerSSHの内部処理のうち,認証 (auth) とSSHセッション管理 (ssh\_session) の各処理は同時接続数に応じて特に増加していくことが分かった.
認証とSSHセッション管理に関しては,ユーザの待ち時間への影響が大きいため,提案手法ではこの時間を短縮する実装上の工夫が必要であると考える.
コンテナ削除処理も接続数に対して若干増加しているが,コンテナ削除処理はユーザが環境を利用し終わった後に行われるため接続時間に大きな影響はないと考える.

%コンテナの削除に関しては環境を利用し終わったあとなので,認証とSSH接続の部分が顕著なので気をつけて設計する必要がある
認証部分に関しては,同時接続数が増えるほど認証サーバの負荷が増えて応答が遅くなっていると考える.
現在の認証サーバはContainerSSHが提供する簡易的なものを利用している.
今後OpenID ConnectやLDAP等のより高度な認証機能を追加していくとより大きなボトルネックになると予想する.
そのため,1度認証したクライアントに対して一時的なトークンを払い出し,以降は一定時間認証サーバにアクセスする必要をなくといったような仕組み等を検討している.

%認証の仕組みを工夫すれば良いかも的な?OAuthでTokenを払い出すみたいなことをしても良いかもしれない,ContainerSSHにある簡易的なものを利用している(並列分散よりも先に認証の仕組みの方で考える)

SSHセッション管理部分に関しては,接続数ごとにSSHの中継をするプロセスが増えていき負荷になっていると想定する.
このため,負荷分散をするための仕組みが必要になると考える.
%スレッドが増えれば増えるほど遅くなる(Context Switch)

\section{まとめ}
%abstractよりも短く,future workをつける
%この論文では〜〜をした.この分野では〜〜〜という問題があり,〜〜という実験をしたら〜〜〜になった.今後は〜〜〜をしていく予定

本論文では,利用者の様々な要求に対して柔軟なコンピューティング環境を動的に提供するためのコンテナ型仮想化環境を提案し,これを実現するためのアーキテクチャについて,既存実装を踏まえて検討および検証した.

今後,特に時間を要する認証とSSHのProxy部分に関して認証サーバへの負荷が少ない認証の仕組み作りや,SSHのProxyを負荷分散するなどの対策を行なっていく予定である.

%\bibliographystyle{ipsjunsrt}
\bibliographystyle{unsrt}
\bibliography{export}

\end{document}